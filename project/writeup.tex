% !TEX TS-program = pdflatex
% !TEX encoding = UTF-8 Unicode

% This is a simple template for a LaTeX document using the "article" class.
% See "book", "report", "letter" for other types of document.

\documentclass[11pt]{article} % use larger type; default would be 10pt

\usepackage[utf8]{inputenc} % set input encoding (not needed with XeLaTeX)

%%% Examples of Article customizations
% These packages are optional, depending whether you want the features they provide.
% See the LaTeX Companion or other references for full information.

%%% PAGE DIMENSIONS
\usepackage{geometry} % to change the page dimensions
\geometry{letterpaper} % or letterpaper (US) or a5paper or....
% \geometry{margin=2in} % for example, change the margins to 2 inches all round
% \geometry{landscape} % set up the page for landscape
%   read geometry.pdf for detailed page layout information

\usepackage{graphicx} % support the \includegraphics command and options

% \usepackage[parfill]{parskip} % Activate to begin paragraphs with an empty line rather than an indent

%%% PACKAGES
\usepackage{booktabs} % for much better looking tables
\usepackage{array} % for better arrays (eg matrices) in maths
\usepackage{paralist} % very flexible & customisable lists (eg. enumerate/itemize, etc.)
\usepackage{verbatim} % adds environment for commenting out blocks of text & for better verbatim
\usepackage{subfig} % make it possible to include more than one captioned figure/table in a single float
% These packages are all incorporated in the memoir class to one degree or another...

%%% HEADERS & FOOTERS
\usepackage{fancyhdr} % This should be set AFTER setting up the page geometry
\pagestyle{fancy} % options: empty , plain , fancy
\renewcommand{\headrulewidth}{0pt} % customise the layout...
\lhead{}\chead{}\rhead{}
\lfoot{}\cfoot{\thepage}\rfoot{}

%%% SECTION TITLE APPEARANCE
\usepackage{sectsty}
\allsectionsfont{\sffamily\mdseries\upshape} % (See the fntguide.pdf for font help)
% (This matches ConTeXt defaults)

%%% ToC (table of contents) APPEARANCE
\usepackage[nottoc,notlof,notlot]{tocbibind} % Put the bibliography in the ToC
\usepackage[titles,subfigure]{tocloft} % Alter the style of the Table of Contents
\renewcommand{\cftsecfont}{\rmfamily\mdseries\upshape}
\renewcommand{\cftsecpagefont}{\rmfamily\mdseries\upshape} % No bold!

\usepackage{amssymb}

\usepackage{algorithm2e}
%%% END Article customizations

%%% The "real" document content comes below...

\usepackage{graphicx}

\title{Project  Writeup}
\author{Bowei Ma}
%\date{} % Activate to display a given date or no date (if empty),
         % otherwise the current date is printed 

\begin{document}
\maketitle

\noindent
\textbf{\Large{Overall Description of the Project\\}}

The major objective of the final project is to implement a solver for simplified steady-state heat equation. The heat transportation system is described where you are transferring some hot fluid with temperature $T_{h}$ within a pipe. Moreover, in order to keep the exterior of the pipe cool, a series of cold air jets with temperature $T_{c}$ are equally distributed along the pipe and continuously impinge on the pipe surface.

The goal of our solver is to determine the mean temprature within the pipe wall using a periodic portion of the pipe.Since the matrix $A$ formed from the geometry is symmetric positive definite and rather sparse, we choose to implement a solver using Conjuagate Gradiant (CG) method to solve the equation $Ax = b$ iteratively.

\bigskip
\noindent
\textbf{\Large{Description of the CG solver implementation\\}}
\bigskip

The problem could be summarized as solving $Ax = b$ for some sparse matrix $A$ (which is the representation of the underlying heat equation) using CG method. In order to efficiently utilize the OOP design concept of C++ language, we encapsulate the equation solver into two classes:

\begin{itemize}

\item \textbf{SparseMatrix} : A class that contains the structure of a sparse matrix. The \texttt{SparseMatrix} class implement several data members and fundamental operations (such as set the dimension of the matrix, add an non-zero entry into the matrx, etc.).

\item \textbf{HeatEquation2D} : A class that contains the setup and solving procedure of the heat equation system. The \texttt{HeatEquation2D} class utilize the instance of  \texttt{SparseMatrix} class, setup the system using a file that describes the geometry of the pipe, and then solve the linear system $Ax = b$ using CG iteration.

\end{itemize}

\newpage
The pseudocode of the CG algorithm is decribed below:
\bigskip

\begin{algorithm}[H]
  \mathchardef\mhyphen="2D
  \caption{Pseudocode of Conjugate Gradient (CG) Algorithm}
  \SetAlgoLined
  initialize $u_{0}$\;
  $r_{0} = b - A u_{0}$\;
  $2\mhyphen norm(r0) = 2\mhyphen norm(r_{0})$\;
  $p_{0} = r_{0}$\;
  $niter = 0$\;
  \While{ $niter < max\#iteration$}{
    $niter = niter +1$\;
    $\alpha_{n} = (r_{n}^{T} r_{n})/ (p_{n}^{T} A p_{n})$\;
    $u_{n+1} = u_{n} + \alpha_{n}p_{n}$\;
    $r_{n+1} = r_{n} - \alpha_{n}A p_{n}$\;
    $2\mhyphen norm(r) = 2\mhyphen norm(r_{n+1})$\;
    \If{ $2\mhyphen norm(r) / 2\mhyphen norm(r0) < threshold$}{
      break\;
      }
    $\beta_{n} = (r_{n+1}^{T} r_{n+1}) / (r_{n}^{T} r_{n})$\;
    $p_{n+1} = r_{n+1} + \beta_{n} p_{n}$\;
    }

\end{algorithm}

\bigskip
\noindent
\textbf{\Large{User's Guide to Execute the Program}}

\bigskip
The program of the heat equation solver is divided into two parts:

\begin{itemize}

\item The computation of the solver is written in C++ using OOP design. With the convienence of \texttt{makefile} , the compilation procedure is pretty straight forward. Basic usage from command line is:

\smallskip
\texttt{make\\}
\texttt{./main [input filename] [solution\_prefix]}
\bigskip

The solution files are written for the initial guess, the last convergence solution and every 10 iterations. The names of the solution files are in the following pattern:
\begin{center}
\texttt{solution\_prefix + number of iterations.txt\\}
\end{center}
After the execution of the program, a meeage with convergence information would be print on the console.

A sample output of the C++ program is given below:
\smallskip\\
\texttt{\$ ./main input2.txt solution\\
SUCCESS: CG solver converged in 157 iterations.}
\smallskip

\item The postprocessing and visualization of the results are written in Python file \texttt{postprocessing.py}, which would compute the mean temperature within the pipe and visualize the thermal distribution using a pseudocolor plot.

The basic usage of the postprocessing procedure from the command line is:

\smallskip
\texttt{python postprocessing.py [input filename] [solution filename]}
\smallskip

A sample output of the Python program is given below:
\bigskip\\
\texttt{Input file processed: input2.txt\\Mean Temperature: 81.8317\\}
\end{itemize}

\noindent
\textbf{\Large{Example of Visualization of the Heat Distribution}}

\begin{figure}[h]
\includegraphics[width = \textwidth]{figure}
\centering
\caption{Example of a pseudocolor plot of input2.txt}
\end{figure}

\newpage
\noindent
\textbf{\textit{\Large{Reference}}}
\bigskip
\\Nick Henderson, \textit{CME 211: Project Part 1}(2015)\\
Nick Henderson, \textit{CME 211: Project Part 2}(2015)

\end{document}
