% !TEX TS-program = pdflatex
% !TEX encoding = UTF-8 Unicode

% This is a simple template for a LaTeX document using the "article" class.
% See "book", "report", "letter" for other types of document.

\documentclass[11pt]{article} % use larger type; default would be 10pt

\usepackage[utf8]{inputenc} % set input encoding (not needed with XeLaTeX)

%%% Examples of Article customizations
% These packages are optional, depending whether you want the features they provide.
% See the LaTeX Companion or other references for full information.

%%% PAGE DIMENSIONS
\usepackage{geometry} % to change the page dimensions
\geometry{letterpaper} % or letterpaper (US) or a5paper or....
% \geometry{margin=2in} % for example, change the margins to 2 inches all round
% \geometry{landscape} % set up the page for landscape
%   read geometry.pdf for detailed page layout information

\usepackage{graphicx} % support the \includegraphics command and options

% \usepackage[parfill]{parskip} % Activate to begin paragraphs with an empty line rather than an indent

%%% PACKAGES
\usepackage{booktabs} % for much better looking tables
\usepackage{array} % for better arrays (eg matrices) in maths
\usepackage{paralist} % very flexible & customisable lists (eg. enumerate/itemize, etc.)
\usepackage{verbatim} % adds environment for commenting out blocks of text & for better verbatim
\usepackage{subfig} % make it possible to include more than one captioned figure/table in a single float
% These packages are all incorporated in the memoir class to one degree or another...

%%% HEADERS & FOOTERS
\usepackage{fancyhdr} % This should be set AFTER setting up the page geometry
\pagestyle{fancy} % options: empty , plain , fancy
\renewcommand{\headrulewidth}{0pt} % customise the layout...
\lhead{}\chead{}\rhead{}
\lfoot{}\cfoot{\thepage}\rfoot{}

%%% SECTION TITLE APPEARANCE
\usepackage{sectsty}
\allsectionsfont{\sffamily\mdseries\upshape} % (See the fntguide.pdf for font help)
% (This matches ConTeXt defaults)

%%% ToC (table of contents) APPEARANCE
\usepackage[nottoc,notlof,notlot]{tocbibind} % Put the bibliography in the ToC
\usepackage[titles,subfigure]{tocloft} % Alter the style of the Table of Contents
\renewcommand{\cftsecfont}{\rmfamily\mdseries\upshape}
\renewcommand{\cftsecpagefont}{\rmfamily\mdseries\upshape} % No bold!

\usepackage{amssymb}

\usepackage{algorithm2e}
%%% END Article customizations

%%% The "real" document content comes below...



\title{Project Part 1 Writeup}
\author{Bowei Ma}
%\date{} % Activate to display a given date or no date (if empty),
         % otherwise the current date is printed 

\begin{document}
\maketitle

\noindent
Pseudocode for the CG algorithm is shown below.

\begin{algorithm}[H]
  \SetAlgoLined
  initialize $u_{0}$\;
  $r_{0} = b - A u_{0}$\;
  $L2normr0 = L2norm(r_{0})$\;
  $p_{0} = r_{0}$\;
  $niter = 0$\;
  \While{ $niter < nitermax$}{
    $niter = niter +1$\;
    $alpha = (r_{n}^{T} r_{n})/ (p_{n}^{T} A p_{n})$\;
    $u_{n+1} = u_{n} + alpha_{n}p_{n}$\;
    $r_{n+1} = r_{n} - alpha_{n}A p_{n}$\;
    $L2normr = L2norm(r_{n+1})$\;
    \If{ $L2normr/L2normr0 < threshold$}{
      break\;
      }
    $beta_{n} = (r_{n+1}^{T} r_{n+1}) / (r_{n}^{T} r_{n})$\;
    $p_{n+1} = r_{n+1} + beta_{n} p_{n}$\;
    }

\end{algorithm}

\bigskip
\noindent
\textbf{Question}: Short discussion of how the CG solver is implemented in terms of functions to eliminate redundant code.

\smallskip
\noindent
\textbf{Answer}:
\smallskip

The CG algorithm is one of the fundamental iterative algorithms to solve Ax = b when A is regarded as a sparse matrix with initial guess x.

In order to wrap operations into functions to eliminate redundancy, I first noiced that the the algorithm involves the following basic matrix and vector operations: 
\begin{itemize}
\item Addition(including sunstraction) of vectors
\item Scalar multiplication of vectors
\item Inner product of two vectors
\item 2-norm computation of a vector
\item Sparse matrix-vector multiplication
\end{itemize}

In addition to notice that the vector addition always appears as x + by as in this algorithm, I implemented the above-mentioned operations into 4 C++ functions, which are:
\begin{itemize}
\item vec\_add\_with\_coeff(x, y, b), which returns x +by
\item vec\_dot\_product(x, y), which computes the inner product of x and y, i.e., $(x^{T} y)$
\item norm(x), which computes the norm of x using inner product and square root
\item csr\_mat\_vec\_product(A, x), which computes the matrix-vector multiplication Ax with A in CSR format
\end{itemize}

\end{document}
